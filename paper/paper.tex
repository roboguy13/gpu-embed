%%
%% This is file `sample-acmsmall.tex',
%% generated with the docstrip utility.
%%
%% The original source files were:
%%
%% samples.dtx  (with options: `acmsmall')
%% 
%% IMPORTANT NOTICE:
%% 
%% For the copyright see the source file.
%% 
%% Any modified versions of this file must be renamed
%% with new filenames distinct from sample-acmsmall.tex.
%% 
%% For distribution of the original source see the terms
%% for copying and modification in the file samples.dtx.
%% 
%% This generated file may be distributed as long as the
%% original source files, as listed above, are part of the
%% same distribution. (The sources need not necessarily be
%% in the same archive or directory.)
%%
%% The first command in your LaTeX source must be the \documentclass command.
\documentclass[acmsmall]{acmart}
\usepackage{booktabs}
\usepackage{listings}

\lstset{language=haskell, deletekeywords={abs}}

\newcommand{\ttt}{\texttt}

%%
%% \BibTeX command to typeset BibTeX logo in the docs
\AtBeginDocument{%
  \providecommand\BibTeX{{%
    \normalfont B\kern-0.5em{\scshape i\kern-0.25em b}\kern-0.8em\TeX}}}

%% Rights management information.  This information is sent to you
%% when you complete the rights form.  These commands have SAMPLE
%% values in them; it is your responsibility as an author to replace
%% the commands and values with those provided to you when you
%% complete the rights form.
\setcopyright{acmcopyright}
\copyrightyear{2020}
\acmYear{2020}
\acmDOI{1}


%%
%% These commands are for a JOURNAL article.
\acmJournal{JACM}
\acmVolume{1}
\acmNumber{1}
\acmArticle{1}
\acmMonth{1}

%%
%% Submission ID.
%% Use this when submitting an article to a sponsored event. You'll
%% receive a unique submission ID from the organizers
%% of the event, and this ID should be used as the parameter to this command.
%%\acmSubmissionID{123-A56-BU3}

%%
%% The majority of ACM publications use numbered citations and
%% references.  The command \citestyle{authoryear} switches to the
%% "author year" style.
%%
%% If you are preparing content for an event
%% sponsored by ACM SIGGRAPH, you must use the "author year" style of
%% citations and references.
%% Uncommenting
%% the next command will enable that style.
%%\citestyle{acmauthoryear}

%%
%% end of the preamble, start of the body of the document source.
\begin{document}

%%
%% The "title" command has an optional parameter,
%% allowing the author to define a "short title" to be used in page headers.
\title{Title Goes Here}

%%
%% The "author" command and its associated commands are used to define
%% the authors and their affiliations.
%% Of note is the shared affiliation of the first two authors, and the
%% "authornote" and "authornotemark" commands
%% used to denote shared contribution to the research.
\author{David Young}
\email{d063y800@ku.edu}

% \authornote{Both authors contributed equally to this research.}
% \email{trovato@corporation.com}
% \orcid{1234-5678-9012}

\author{Andrew Gill}
\email{andygill@ku.edu}

% \authornotemark[1]
% \email{webmaster@marysville-ohio.com}
% \affiliation{%
%   \institution{Institute for Clarity in Documentation}
%   \streetaddress{P.O. Box 1212}
%   \city{Dublin}
%   \state{Ohio}
%   \postcode{43017-6221}
% }

% \author{Lars Th{\o}rv{\"a}ld}
% \affiliation{%
%   \institution{The Th{\o}rv{\"a}ld Group}
%   \streetaddress{1 Th{\o}rv{\"a}ld Circle}
%   \city{Hekla}
%   \country{Iceland}}
% \email{larst@affiliation.org}

% \author{Valerie B\'eranger}
% \affiliation{%
%   \institution{Inria Paris-Rocquencourt}
%   \city{Rocquencourt}
%   \country{France}
% }

% \author{Aparna Patel}
% \affiliation{%
%  \institution{Rajiv Gandhi University}
%  \streetaddress{Rono-Hills}
%  \city{Doimukh}
%  \state{Arunachal Pradesh}
%  \country{India}}

% \author{Huifen Chan}
% \affiliation{%
%   \institution{Tsinghua University}
%   \streetaddress{30 Shuangqing Rd}
%   \city{Haidian Qu}
%   \state{Beijing Shi}
%   \country{China}}

% \author{Charles Palmer}
% \affiliation{%
%   \institution{Palmer Research Laboratories}
%   \streetaddress{8600 Datapoint Drive}
%   \city{San Antonio}
%   \state{Texas}
%   \postcode{78229}}
% \email{cpalmer@prl.com}

% \author{John Smith}
% \affiliation{\institution{The Th{\o}rv{\"a}ld Group}}
% \email{jsmith@affiliation.org}

% \author{Julius P. Kumquat}
% \affiliation{\institution{The Kumquat Consortium}}
% \email{jpkumquat@consortium.net}

%%
%% By default, the full list of authors will be used in the page
%% headers. Often, this list is too long, and will overlap
%% other information printed in the page headers. This command allows
%% the author to define a more concise list
%% of authors' names for this purpose.
% \renewcommand{\shortauthors}{Trovato and Tobin, et al.}

%%
%% The abstract is a short summary of the work to be presented in the
%% article.
\begin{abstract}
  Domain specific languages (DSLs) provide a powerful tool for to abstract over
  a class of problems. In this paper, we introduce a technique for representing
  algebraic data types and pattern matching in a DSL embedded in Haskell. This
  representation is automated through the use of GHC Generics and Template Haskell.
\end{abstract}

\begin{CCSXML}
<ccs2012>
   <concept>
       <concept_id>10011007.10011006.10011050.10011017</concept_id>
       <concept_desc>Software and its engineering~Domain specific languages</concept_desc>
       <concept_significance>500</concept_significance>
       </concept>
 </ccs2012>
\end{CCSXML}

\ccsdesc[500]{Software and its engineering~Domain specific languages}

% \keywords{domain specific languages}

%%
%% The code below is generated by the tool at http://dl.acm.org/ccs.cfm.
%% Please copy and paste the code instead of the example below.
%%
% \begin{CCSXML}
% <ccs2012>
%  <concept>
%   <concept_id>10010520.10010553.10010562</concept_id>
%   <concept_desc>Computer systems organization~Embedded systems</concept_desc>
%   <concept_significance>500</concept_significance>
%  </concept>
%  <concept>
%   <concept_id>10010520.10010575.10010755</concept_id>
%   <concept_desc>Computer systems organization~Redundancy</concept_desc>
%   <concept_significance>300</concept_significance>
%  </concept>
%  <concept>
%   <concept_id>10010520.10010553.10010554</concept_id>
%   <concept_desc>Computer systems organization~Robotics</concept_desc>
%   <concept_significance>100</concept_significance>
%  </concept>
%  <concept>
%   <concept_id>10003033.10003083.10003095</concept_id>
%   <concept_desc>Networks~Network reliability</concept_desc>
%   <concept_significance>100</concept_significance>
%  </concept>
% </ccs2012>
% \end{CCSXML}

% \ccsdesc[500]{Computer systems organization~Embedded systems}
% \ccsdesc[300]{Computer systems organization~Redundancy}
% \ccsdesc{Computer systems organization~Robotics}
% \ccsdesc[100]{Networks~Network reliability}

%%
%% Keywords. The author(s) should pick words that accurately describe
%% the work being presented. Separate the keywords with commas.
% \keywords{datasets, neural networks, gaze detection, text tagging}


%%
%% This command processes the author and affiliation and title
%% information and builds the first part of the formatted document.
\maketitle

\section{Introduction}

Embedded domain specific languages (EDSLs) have long been a useful and effective
technique for constructing reusable tools for writing programs for a variety of
different problem domains. Haskell is a language which is particularly
well-suited to EDSLs due to its lazy evaluation, first-class functions and
lexical closures.

\subsection{Translation Example} % TODO: Does it make sense for this to be a
                                 % subsection of the intro?

In the example below, \ttt{example} would get transformed into \ttt{example'}:

\begin{lstlisting}
import Data.Char (ord)

x :: Either Char Int
x = Left 'a'

example :: Int
example =
  case x of
    Left  c -> ord c
    Right i -> i

example' :: Int
example' =
  abs (CaseExp (rep x)
               (SumMatch (OneProdMatch (\c -> ord c))
                         (OneSumMatch
                           (OneProdMatch (\i -> i)))))
\end{lstlisting}

[\ldots]

\section{Representing Algebraic Datatypes}

\subsection{GHC Generics \ttt{Rep}}

[Remove metadata from Generic representation and turn tree-like structure into a (nested) list-like structure]

\subsection{\ttt{E}, \ttt{ERep}, \ttt{ERepTy}}

There are three interconnected foundational parts: \ttt{E}, \ttt{ERep} and
\ttt{ERepTy}. \ttt{E} is the deep embedding of the EDSL (a GADT that encodes
expressions in the DSL language). \ttt{ERep} is a type class which represents
all Haskell types which can be represented in the DSL. \ttt{ERepTy} is a type
family associated to the \ttt{ERep} type class, which represents a "canonical form"
of the given type. This canonical form can be immediately constructed in the EDSL.
Canonical form types crucially include \ttt{Either} and \ttt{(,)}, which
allow all combinations of basic sum types and product types to be encoded. More
information on this encoding is given in [a different section].

This information is brought into the \ttt{E} type via the constructor
\ttt{Repped}. The \ttt{E} also contains constructors representing \ttt{Either}
and \ttt{(,)} values:

\begin{lstlisting}
data E t where
  ...
  Repped :: ERep t => ERepTy t -> E t

  LeftExp :: E a -> E (Either a b)
  RightExp :: E b -> E (Either a b)

  PairExp :: E a -> E b -> E (a, b)
  ...
\end{lstlisting}

\ttt{ERep} and \ttt{ERepTy} provide an interface for transferring values between the EDSL
world and the Haskell world:

\begin{lstlisting}
class ERep t where
  type ERepTy t

  rep :: t -> E t
  rep = Repped . rep'

  unrep :: ERepTy t -> t

  rep' :: t -> ERepTy t
\end{lstlisting}

The key algebraic instances mentioned before are as follows:

\begin{lstlisting}
instance (ERep a, ERep b) => ERep (Either a b) where
  type ERepTy (Either a b) = Either a b

  rep (Left x)  = LeftExp (rep x)
  rep (Right y) = RightExp (rep y)

  unrep = id
  rep'  = id

instance (ERep a, ERep b) => ERep (a, b) where
  type ERepTy (a, b) = (a, b)

  rep (x, y) = PairExp (rep x) (rep y)
  unrep = id
  rep'  = id
\end{lstlisting}

\section{Representing pattern matches}

Within the \ttt{E} expression language, a pattern match is represented by the
\ttt{CaseExp} constructor:

\begin{lstlisting}
data E t where
  ...
  CaseExp :: ERep t => E t -> SumMatch (ERepTy t) r -> E r
  ...
\end{lstlisting}

A value of type \ttt{SumMatch a b} represents a computation within the EDSL
which destructures a value of type \ttt{E a} and produces a value of type \ttt{E b}.
Therefore, \ttt{SumMatch (ERepTy t) r} represents a computation which destructures
a value of type \ttt{E (ERepTy t)} and produces a value of type \ttt{E b}.



\subsection{\ttt{SumMatch}}

The overall structure of \ttt{SumMatch} is a (heterogeneous) list of \ttt{ProdMatch} values.
Each item of this list corresponds exactly to one branch in the original \ttt{case} match.

\begin{lstlisting}
data SumMatch s t where
  SumMach ::
    (ERep a, ERep b, ERepTy b ~ b)
      => ProdMatch a r -> SumMatch b r -> SumMatch (Either a b) r

  EmptyMatch :: SumMatch Void r

  OneSumMatch ::
    (ERep a, ERep b, ERepTy a ~ a)
      => ProdMatch a b -> SumMatch a b
\end{lstlisting}

Note the \ttt{ERepTy a ~ a} constraints. This constraint ensures that the type \ttt{a}
is already in canonical form (that is, consists entirely of \ttt{Either}, \ttt{(,)} and
base types).

\subsection{\ttt{ProdMatch}}

\ttt{ProdMatch s t} is equivalent to a function a curried function from \ttt{s} to \ttt{t}.
Note that \ttt{s} will be a (potentially nested) pair type. For example,
\ttt{ProdMatch (a, (b, c)) r} will be equivalent to \ttt{a -> b -> c -> r}.

\begin{lstlisting}
data ProdMatch s t where
  ProdMatch ::
    (ERep a, ERep b)
      => (E a -> ProdMatch b r) -> ProdMatch (a, b) r

  OneProdMatch :: ERep a => (E a -> E r) -> ProdMatch a r

  NullaryMatch :: ERep r => E r -> ProdMatch a r
\end{lstlisting}

\section{Recovering standard semantics for pattern matches}

\begin{lstlisting}
sumMatchAbs :: ERep s => SumMatch (ERepTy s) t -> s -> t
...

prodMatchAbs :: ERep s => ProdMatch s t -> s -> t
...

abs :: E t -> t
abs (CaseExp x f) = sumMatchAbs f (abs x)

abs (LeftExp x) = Left (abs x)
abs (RightExp y) = Right (abs y)
abs (PairExp x y) = (abs x, abs y)

abs (Repped x) = unrep x
...
\end{lstlisting}

\section{Constructing values}

\begin{lstlisting}
construct ::
  Construct (a -> b)
    => (a -> b) -> LiftedFn (a -> b)
construct = construct' . Construct

-- Example: LiftedFn (a -> b -> c) is: E a -> E b -> E c
type family LiftedFn t where
  LiftedFn (a -> b) = E a -> LiftedFn b
  LiftedFn b = E b

class Construct t where
  construct' :: E t -> LiftedFn t
\end{lstlisting}

Example GHCi session, demonstrating \ttt{construct}:

\begin{lstlisting}
ghci> :t construct (,)
construct (,)
  :: (ERep a, ERep b) => E a -> E b -> E (a, b)

ghci> :t construct (,,)
construct (,,)
  :: (ERep a, ERep b, ERep c) =>
     E a -> E b -> E c -> E (a, b, c)

ghci> :t construct (,,) (rep False) (rep 3.1 :: E Float) (rep 1 :: E Int)
construct (,,) (rep False) (rep 3.1 :: E Float) (rep 1 :: E Int)
  :: E (Bool, Float, Int)
\end{lstlisting}


\section{Citations and Bibliographies}

% The use of \BibTeX\ for the preparation and formatting of one's
% references is strongly recommended. Authors' names should be complete
% --- use full first names (``Donald E. Knuth'') not initials
% (``D. E. Knuth'') --- and the salient identifying features of a
% reference should be included: title, year, volume, number, pages,
% article DOI, etc.

% The bibliography is included in your source document with these two
% commands, placed just before the \verb|\end{document}| command:
% \begin{verbatim}
%   \bibliographystyle{ACM-Reference-Format}
%   \bibliography{bibfile}
% \end{verbatim}
% where ``\verb|bibfile|'' is the name, without the ``\verb|.bib|''
% suffix, of the \BibTeX\ file.

% Citations and references are numbered by default. A small number of
% ACM publications have citations and references formatted in the
% ``author year'' style; for these exceptions, please include this
% command in the {\bfseries preamble} (before
% ``\verb|\begin{document}|'') of your \LaTeX\ source:
% \begin{verbatim}
%   \citestyle{acmauthoryear}
% \end{verbatim}

%   Some examples.  A paginated journal article \cite{Abril07}, an
%   enumerated journal article \cite{Cohen07}, a reference to an entire
%   issue \cite{JCohen96}, a monograph (whole book) \cite{Kosiur01}, a
%   monograph/whole book in a series (see 2a in spec. document)
%   \cite{Harel79}, a divisible-book such as an anthology or compilation
%   \cite{Editor00} followed by the same example, however we only output
%   the series if the volume number is given \cite{Editor00a} (so
%   Editor00a's series should NOT be present since it has no vol. no.),
%   a chapter in a divisible book \cite{Spector90}, a chapter in a
%   divisible book in a series \cite{Douglass98}, a multi-volume work as
%   book \cite{Knuth97}, an article in a proceedings (of a conference,
%   symposium, workshop for example) (paginated proceedings article)
%   \cite{Andler79}, a proceedings article with all possible elements
%   \cite{Smith10}, an example of an enumerated proceedings article
%   \cite{VanGundy07}, an informally published work \cite{Harel78}, a
%   doctoral dissertation \cite{Clarkson85}, a master's thesis:
%   \cite{anisi03}, an online document / world wide web resource
%   \cite{Thornburg01, Ablamowicz07, Poker06}, a video game (Case 1)
%   \cite{Obama08} and (Case 2) \cite{Novak03} and \cite{Lee05} and
%   (Case 3) a patent \cite{JoeScientist001}, work accepted for
%   publication \cite{rous08}, 'YYYYb'-test for prolific author
%   \cite{SaeediMEJ10} and \cite{SaeediJETC10}. Other cites might
%   contain 'duplicate' DOI and URLs (some SIAM articles)
%   \cite{Kirschmer:2010:AEI:1958016.1958018}. Boris / Barbara Beeton:
%   multi-volume works as books \cite{MR781536} and \cite{MR781537}. A
%   couple of citations with DOIs:
%   \cite{2004:ITE:1009386.1010128,Kirschmer:2010:AEI:1958016.1958018}. Online
%   citations: \cite{TUGInstmem, Thornburg01, CTANacmart}. Artifacts:
%   \cite{R} and \cite{UMassCitations}.

% \section{Acknowledgments}

%Identification of funding sources and other support, and thanks to
%individuals and groups that assisted in the research and the
%preparation of the work should be included in an acknowledgment
%section, which is placed just before the reference section in your
%document.

%This section has a special environment:
%\begin{verbatim}
%  \begin{acks}
%  ...
%  \end{acks}
%\end{verbatim}
%so that the information contained therein can be more easily collected
%during the article metadata extraction phase, and to ensure
%consistency in the spelling of the section heading.

%Authors should not prepare this section as a numbered or unnumbered {\verb|\section|}; please use the ``{\verb|acks|}'' environment.

%\section{Appendices}

%If your work needs an appendix, add it before the
%``\verb|\end{document}|'' command at the conclusion of your source
%document.

%Start the appendix with the ``\verb|appendix|'' command:
%\begin{verbatim}
%  \appendix
%\end{verbatim}
%and note that in the appendix, sections are lettered, not
%numbered. This document has two appendices, demonstrating the section
%and subsection identification method.

%\section{SIGCHI Extended Abstracts}

%The ``\verb|sigchi-a|'' template style (available only in \LaTeX\ and
%not in Word) produces a landscape-orientation formatted article, with
%a wide left margin. Three environments are available for use with the
%``\verb|sigchi-a|'' template style, and produce formatted output in
%the margin:
%\begin{itemize}
%\item {\verb|sidebar|}:  Place formatted text in the margin.
%\item {\verb|marginfigure|}: Place a figure in the margin.
%\item {\verb|margintable|}: Place a table in the margin.
%\end{itemize}

%%%
%%% The acknowledgments section is defined using the "acks" environment
%%% (and NOT an unnumbered section). This ensures the proper
%%% identification of the section in the article metadata, and the
%%% consistent spelling of the heading.
%\begin{acks}
%To Robert, for the bagels and explaining CMYK and color spaces.
%\end{acks}

%%%
%%% The next two lines define the bibliography style to be used, and
%%% the bibliography file.
%\bibliographystyle{ACM-Reference-Format}
%\bibliography{sample-base}

%%%
%%% If your work has an appendix, this is the place to put it.
%\appendix

%\section{Research Methods}

%\subsection{Part One}

%Lorem ipsum dolor sit amet, consectetur adipiscing elit. Morbi
%malesuada, quam in pulvinar varius, metus nunc fermentum urna, id
%sollicitudin purus odio sit amet enim. Aliquam ullamcorper eu ipsum
%vel mollis. Curabitur quis dictum nisl. Phasellus vel semper risus, et
%lacinia dolor. Integer ultricies commodo sem nec semper.

%\subsection{Part Two}

%Etiam commodo feugiat nisl pulvinar pellentesque. Etiam auctor sodales
%ligula, non varius nibh pulvinar semper. Suspendisse nec lectus non
%ipsum convallis congue hendrerit vitae sapien. Donec at laoreet
%eros. Vivamus non purus placerat, scelerisque diam eu, cursus
%ante. Etiam aliquam tortor auctor efficitur mattis.

%\section{Online Resources}

%Nam id fermentum dui. Suspendisse sagittis tortor a nulla mollis, in
%pulvinar ex pretium. Sed interdum orci quis metus euismod, et sagittis
%enim maximus. Vestibulum gravida massa ut felis suscipit
%congue. Quisque mattis elit a risus ultrices commodo venenatis eget
%dui. Etiam sagittis eleifend elementum.

%Nam interdum magna at lectus dignissim, ac dignissim lorem
%rhoncus. Maecenas eu arcu ac neque placerat aliquam. Nunc pulvinar
%massa et mattis lacinia.

\end{document}
\endinput
